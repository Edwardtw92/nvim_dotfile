%%%%%%%%%%%%%%%%%%%%%%%
% Some basic packages %
%%%%%%%%%%%%%%%%%%%%%%%
\usepackage[utf8]{inputenc}
\usepackage[T1]{fontenc}
\usepackage{textcomp}
\usepackage{url}
\usepackage{float}
\usepackage{booktabs}
\usepackage{enumitem}
%%%%%%%%%%%%%%%%%%%%%%%


%%%%%%%%
% Date %
%%%%%%%%
\usepackage{datetime2}
%%%%%%%%


%%%%%%%%%%%%%%
% Scripted r %
%%%%%%%%%%%%%%
\usepackage{calligra}
\DeclareMathAlphabet{\mathcalligra}{T1}{calligra}{m}{n}
\DeclareFontShape{T1}{calligra}{m}{n}{<->s*[2.2]callig15}{}
\newcommand{\scriptr}{\mathcalligra{r}\,}
\newcommand{\boldscriptr}{\pmb{\mathcalligra{r}}\,}
% \allowdisplaybreaks
%%%%%%%%%%%%%%


%%%%%%%%%%%%%%%%%%%%%%%%%%%%%%%%%%%%%%%%%%%%%%%%%%%%%%%%%%
% Don't indent paragraphs, leave some space between them %
%%%%%%%%%%%%%%%%%%%%%%%%%%%%%%%%%%%%%%%%%%%%%%%%%%%%%%%%%%
\usepackage{parskip}
%%%%%%%%%%%%%%%%%%%%%%%%%%%%%%%%%%%%%%%%%%%%%%%%%%%%%%%%%%


%%%%%%%%%%%%%%%%%%%%%$$
% Easy quotation mark %
%%%%%%%%%%%%%%%%%%%%%%%
\usepackage{csquotes}
%%%%%%%%%%%%%%%%%%%%%%%


%%%%%%%%%%%%%%%%%%%%%%%%%%%%%%%%%%%%%%%
% Hide page number when page is empty %
%%%%%%%%%%%%%%%%%%%%%%%%%%%%%%%%%%%%%%%
\usepackage{emptypage}
\usepackage{subcaption}
\usepackage{multicol}
\usepackage{xcolor}
%%%%%%%%%%%%%%%%%%%%%%%%%%%%%%%%%%%%%%%

%%%%%%%%
% Math %
%%%%%%%%
\usepackage{amsmath, amsfonts, mathtools, amsthm, amssymb}
\usepackage{physics}
\usepackage{dsfont}

% Fancy script capitlas
\usepackage{mathrsfs}
\usepackage{cancel}

% Bold math
\usepackage{bm}

% System of equations
\usepackage{systeme}
%%%%%%%%

%%%%%%%%%%%%%%%%%%
% Other packages %
%%%%%%%%%%%%%%%%%%
\usepackage{nicefrac,fancyhdr,hyperref,graphicx,adjustbox}
\hypersetup{colorlinks=true,urlcolor=blue,citecolor=blue,linkcolor=blue}
% \usepackage[left=1.5cm, right=1.5cm, top=1.3cm, includehead, includefoot]{geometry}
% \usepackage[left=1.0cm, right=1.0cm, top=1.0cm, bottom= 1.0cm]{geometry}
\usepackage[d]{esvect}
\usepackage{tikz}
\usepackage{esint}
\usepackage{titlesec}
\setcounter{secnumdepth}{4}
\usepackage{setspace}
% \doublespacin
%%%%%%%%%%%%%%%%%%%%%%%



%%%%%%%%%%%%%%
%% commands %%
%%%%%%%%%%%%%%

%%%%%%%%
% Sets %
%%%%%%%%
\newcommand{\C}{\ensuremath{\mathbb{C}}}
\newcommand{\N}{\ensuremath{\mathbb{N}}}
\newcommand{\Q}{\ensuremath{\mathbb{Q}}}
\newcommand{\R}{\ensuremath{\mathbb{R}}}
\newcommand{\Z}{\ensuremath{\mathbb{Z}}}
\newcommand{\Zm}{\Z_m}
\newcommand{\Zp}{\Z_p}
\newcommand{\F}{\ensuremath{\mathbb{F}}}
% \newcommand{\O}{\ensuremath{\emptyset}}
%%%%%%%%


%%%%%%%%%%
% logics %
%%%%%%%%%%
\newcommand{\xor}{\veebar}
\newcommand{\nand}{\uparrow}
%%%%%%%%%%


%%%%%%%%%%%%%%%%%%%%%%%%%%%%%%%%%%%%%%
% Make implies and impliedby shorter %
%%%%%%%%%%%%%%%%%%%%%%%%%%%%%%%%%%%%%%
\let\implies\Rightarrow
\let\impliedby\Leftarrow
\let\iff\Leftrightarrow
% \let\epsilon\varepsilon
%%%%%%%%%%%%%%%%%%%%%%%%%%%%%%%%%%%%%%


%%%%%%%%%%%%%%%%%%%%%%%%%%%%%%%%%%%%%%%%%%%%%%
% Add \contra symbol to denote contradiction %
%%%%%%%%%%%%%%%%%%%%%%%%%%%%%%%%%%%%%%%%%%%%%%
\usepackage{stmaryrd} % for \lightning
\newcommand\contra{\scalebox{1.5}{$\lightning$}}
%%%%%%%%%%%%%%%%%%%%%%%%%%%%%%%%%%%%%%%%%%%%%%


%%%%%%%%%%%%%%%%%%%%%%%%%%%%%%%%%
% Command for short corrections %
%%%%%%%%%%%%%%%%%%%%%%%%%%%%%%%%%
% Usage: 1+1=\correct{3}{2}
\definecolor{correct}{HTML}{009900}
\newcommand\correct[2]{\ensuremath{\:}{\color{red}{#1}}\ensuremath{\to }{\color{correct}{#2}}\ensuremath{\:}}
\newcommand\green[1]{{\color{correct}{#1}}}
%%%%%%%%%%%%%%%%%%%%%%%%%%%%%%%%%


%%%%%%%%%%%%%%%%%%%
% horizontal rule %
%%%%%%%%%%%%%%%%%%%
\newcommand\hr{
    \noindent\rule[0.5ex]{\linewidth}{0.5pt}
}
%%%%%%%%%%%%%%%%%%%


%%%%%%%%%%%%%%
% hide parts %
%%%%%%%%%%%%%%
\newcommand\hide[1]{}
%%%%%%%%%%%%%%


%%%%%%%%%
% bases %
%%%%%%%%%
\newcommand{\mA}{\mathcal{A}}
\newcommand{\mB}{\mathcal{B}}
\newcommand{\mC}{\mathcal{C}}
\newcommand{\mD}{\mathcal{D}}
\newcommand{\mE}{\mathcal{E}}
\newcommand{\mL}{\mathcal{L}}
\newcommand{\mM}{\mathcal{M}}
\newcommand{\mO}{\mathcal{O}}
\newcommand{\mP}{\mathcal{P}}
\newcommand{\mS}{\mathcal{S}}
\newcommand{\mT}{\mathcal{T}}
%%%%%%%%%


%%%%%%%%%%%%%%%%%%
% linear algebra %
%%%%%%%%%%%%%%%%%%
\newcommand{\diag}{\operatorname{diag}}
\newcommand{\adj}{\operatorname{adj}}
% \newcommand{\rank}{\operatorname{rank}}
\newcommand{\spn}{\operatorname{Span}}
\newcommand{\proj}{\operatorname{proj}}
\newcommand{\prp}{\operatorname{perp}}
\newcommand{\refl}{\operatorname{refl}}
% \newcommand{\tr}{\operatorname{tr}}
\newcommand{\nul}{\operatorname{Null}}
\newcommand{\nully}{\operatorname{nullity}}
\newcommand{\range}{\operatorname{Range}}
\renewcommand{\ker}{\operatorname{Ker}}
\newcommand{\col}{\operatorname{Col}}
\newcommand{\row}{\operatorname{Row}}
\newcommand{\cof}{\operatorname{cof}}
\newcommand{\Num}{\operatorname{Num}}
\newcommand{\Id}{\operatorname{Id}}
\newcommand{\id}{\operatorname{id}}
\newcommand{\ipb}{\langle \thinspace, \rangle}
%\newcommand{\ip}[2]{\left\langle #1, #2\right\rangle} % inner products
\newcommand{\M}[2]{M_{#1\times #2}(\F)}
\newcommand{\RREF}{\operatorname{RREF}}
\newcommand{\REF}{\operatorname{REF}}
\newcommand{\cv}[1]{\begin{bmatrix} #1 \end{bmatrix}}
%\newenvironment{amatrix}[1]{\left[\begin{array}{@{}*{\numexpr#1-1}{c}|c@{}}}{\end{array}\right]}
\newcommand{\am}[2]{\begin{amatrix}{#1} #2 \end{amatrix}}
%%%%%%%%%%%%%%%%%%


%%%%%%%%%%%
% vectors %
%%%%%%%%%%%
\newcommand{\vzero}{\vv{0}}
%\newcommand{\va}{\vv{a}}
%\newcommand{\vb}{\vv{b}}
\newcommand{\vc}{\vv{c}}
\newcommand{\vd}{\vv{d}}
\newcommand{\ve}{\vv{e}}
\newcommand{\vf}{\vv{f}}
\newcommand{\vg}{\vv{g}}
\newcommand{\vh}{\vv{h}}
\newcommand{\vl}{\vv{\ell}}
\newcommand{\vm}{\vv{m}}
\newcommand{\vn}{\vv{n}}
\newcommand{\vp}{\vv{p}}
\newcommand{\vq}{\vv{q}}
\newcommand{\vr}{\vv{r}}
\newcommand{\vs}{\vv{s}}
\newcommand{\vt}{\vv{t}}
%\newcommand{\vu}{\vv{u}}
\newcommand{\vvv}{{\vv{v}}}
\newcommand{\vw}{\vv{w}}
\newcommand{\vx}{\vv{x}}
\newcommand{\vy}{\vv{y}}
\newcommand{\vz}{\vv{z}}
%%%%%%%%%%%


%%%%%%%%%%%
% display %
%%%%%%%%%%%
\newcommand{\ds}{\displaystyle}
%\newcommand{\qand}{\quad\text{and}}
\newcommand{\qandq}{\quad\text{and}\quad}
\newcommand{\hint}{\textbf{Hint: }}
\newcommand*\circled[1]{\tikz[baseline=(char.base)]{
		\node[shape=circle,draw,inner sep=2pt] (char) {#1};}}
%%%%%%%%%%%


%%%%%%%%
% misc %
%%%%%%%%
\newcommand{\area}{\operatorname{area}}
\newcommand{\vol}{\operatorname{vol}}
\newcommand{\red}[1]{{\color{red} #1}}
\newcommand{\rc}{\red{\checkmark}}
%%%%%%%%


%%%%%%%%%
% other %
%%%%%%%%%
%\newcommand{\bd}[2]{\begin{tcolorbox}[title={Definition: #1}, colback=red!5,colframe=red!75!black] #2 \end{tcolorbox} \hfill \break}
\newcommand{\bp}[2]{\begin{tcolorbox}[title={Proposition: #1}, colback=blue!5,colframe=blue!75!black] #2 \end{tcolorbox} \hfill \break}
\newcommand{\bt}[2]{\begin{tcolorbox}[title={Theorem: #1}, colback=blue!5,colframe=blue!75!black] #2 \end{tcolorbox} \hfill \break}
\newcommand{\bc}[2]{\begin{tcolorbox}[title={Corollary: #1}, colback=blue!5,colframe=blue!75!black] #2 \end{tcolorbox} \hfill \break}
%\newcommand{\dv}[1]{\vv{#1}=\cv{#1 _1\\ \vdots\\ #1 _n}}
\newcommand{\ti}[1]{\textit{#1}}
%%%%%%%%%
\newcommand{\normalisation}[1]{\frac{\vv{#1}}{\norm{\vv{#1}}}}
% \newcommand{\q}{\begin{flushright} Q.E.D.\end{flushright}}
% \newcommand{\projv}[2]{\proj_{\vv{#1}}\left(\vv{#2}\right)}
% \newcommand{\prpv}[2]{\prp_{\vv{#1}}(\vv{#2})}
\newcommand{\ca}[1]{\begin{cases}#1\end{cases}}
\newcommand{\st}{\text{ s.t. }}
\newcommand{\rowm}[1]{\row(#1)}
\newcommand{\nulm}[1]{\nul(#1)}
\newcommand{\balign}[1]{\begin{align*}#1\end{align*}}
\newcommand{\sm}[3]{\begin{amatrix}{#1}{#2} #3 \end{amatrix}}
\newcommand{\smt}[1]{\begin{smatrixt} #1 \end{smatrixt}}
\newcommand{\cA}[1]{\cae{A}{#1}}
\newcommand{\elam}[1]{E_{\lam}(#1)}
\newcommand{\degg}[1]{\deg(#1)}

% algebra
\newcommand{\ncr}[2]{\binom{#1}{#2}}
\newcommand{\gcdd}[2]{\gcd(#1, #2)}
\DeclareMathOperator{\lcd}{ord}
\newcommand{\modd}[1]{\pmod{#1}}
\newcommand{\con}{\equiv}
\newcommand{\ncon}{\not\equiv}
\DeclareMathOperator{\lcm}{lcm}
\DeclareMathOperator{\sgn}{sgn}

% group theory
\DeclareMathOperator{\Perm}{Perm}
\DeclareMathOperator{\ord}{ord}
\DeclareMathOperator{\Cl}{Cl}
\DeclareMathOperator{\Hom}{Hom}
\DeclareMathOperator{\Iso}{Iso}
\DeclareMathOperator{\Ker}{Ker}
\DeclareMathOperator{\Image}{Image}
\DeclareMathOperator{\End}{End}
\DeclareMathOperator{\Aut}{Aut}
\DeclareMathOperator{\Inn}{Inn}

% quantum theory
\newcommand{\I}{\mathds{1}}
\DeclareMathAlphabet{\mathbbold}{U}{bbold}{m}{n}
\newcommand{\zero}{\mathbbold{0}}
\DeclareMathOperator{\spec}{spec}


\newenvironment{amatrix}[1]{%
	\left[\begin{array}{@{}*{#1}{c}|c@{}}
	}{%
	\end{array}\right]
}

\newenvironment{smatrix}[2]{%
	\left[\begin{array}{@{}*{#1}{c}|@{}*{#2}{c}}
	}{%
	\end{array}\right]
}

\newenvironment{smatrixt}{%
	\left[\begin{array}{ccc|ccc}
	}{%
	\end{array}\right]
}

% \setlength\parindent{0pt}

%\setlength{\parskip}{0.75\baselineskip plus 2pt}
%\usepackage[skip=0.75\baselineskip plus 2pt]{parskip}
%\setlength\parskip{12pt plus 1em}


% DEC
\newcommand{\Bd}[1]{\mbox{Bd}(#1)}
\newcommand{\Int}[1]{\mbox{Int}(#1)}
\newcommand{\Ext}[1]{\mbox{Ext}(#1)}


% Environments
\makeatother
% For box around Definition, Theorem, \ldots
\usepackage{mdframed}
\mdfsetup{skipabove=1em,skipbelow=0em}
\theoremstyle{definition}
\newmdtheoremenv[nobreak=true]{definitie}{Definitie}
\newmdtheoremenv[nobreak=true]{eigenschap}{Eigenschap}
\newmdtheoremenv[nobreak=true]{gevolg}{Gevolg}
\newmdtheoremenv[nobreak=true]{lemma}{Lemma}
\newmdtheoremenv[nobreak=true]{propositie}{Propositie}
\newmdtheoremenv[nobreak=true]{stelling}{Stelling}
\newmdtheoremenv[nobreak=true]{wet}{Wet}
\newmdtheoremenv[nobreak=true]{postulaat}{Postulaat}
\newmdtheoremenv{conclusie}{Conclusie}
\newmdtheoremenv{toemaatje}{Toemaatje}
\newmdtheoremenv{vermoeden}{Vermoeden}
\newtheorem*{herhaling}{Herhaling}
\newtheorem*{intermezzo}{Intermezzo}
\newtheorem*{notatie}{Notatie}
\newtheorem*{observatie}{Observatie}
\newtheorem*{oef}{Oefening}
\newtheorem*{opmerking}{Opmerking}
\newtheorem*{praktisch}{Praktisch}
\newtheorem*{probleem}{Probleem}
\newtheorem*{terminologie}{Terminologie}
\newtheorem*{toepassing}{Toepassing}
\newtheorem*{uovt}{UOVT}
% \newtheorem*{vb}{Voorbeeld}
\newtheorem*{vraag}{Vraag}

\newmdtheoremenv[nobreak=true]{definition}{Definition}
\newtheorem*{eg}{Example}
\newtheorem*{notation}{Notation}
\newtheorem*{previouslyseen}{As previously seen}
\newtheorem*{remark}{Remark}
\newtheorem*{note}{Note}
\newtheorem*{problem}{Problem}
\newtheorem*{observe}{Observe}
\newtheorem*{property}{Property}
\newtheorem*{intuition}{Intuition}
% \newmdtheoremenv[nobreak=true]{prop}{Proposition}
\newmdtheoremenv[nobreak=true]{theorem}{Theorem}
\newmdtheoremenv[nobreak=true]{corollary}{Corollary}

% End example and intermezzo environments with a small diamond (just like proof
% environments end with a small square)
\usepackage{etoolbox}
\AtEndEnvironment{vb}{\null\hfill$\diamond$}%
\AtEndEnvironment{intermezzo}{\null\hfill$\diamond$}%
% \AtEndEnvironment{opmerking}{\null\hfill$\diamond$}%

% Fix some spacing
% http://tex.stackexchange.com/questions/22119/how-can-i-change-the-spacing-before-theorems-with-amsthm
\makeatletter
\def\thm@space@setup{%
  \thm@preskip=\parskip \thm@postskip=0pt
}


%======================================================================
% Todonotes and inline notes in fancy boxes
\usepackage{todonotes}
\usepackage{tcolorbox}

% Make boxes breakable
\tcbuselibrary{breakable}

% def
\newenvironment{defn}[1]{\begin{tcolorbox}[
    arc=0mm,
    colback=white,
    colframe=blue!60!black,
    title={Def: #1},
    fonttitle=\sffamily,
    breakable
]}{\end{tcolorbox}}

\newenvironment{defm}[1]{\begin{tcolorbox}[
    arc=0mm,
    % colback=white,
    % colback=yellow!15!white,
    colback=blue!8!white,
    colframe=blue!60!black,
    title={Def: #1},
    fonttitle=\sffamily,
    breakable
]}{\end{tcolorbox}}

\definecolor{purple}{rgb}{76, 0, 153}
\newenvironment{thm}[1]{\begin{tcolorbox}[
    arc=0mm,
    colback=white,
    colframe=purple!60!black,
    title={Theorem: #1},
    fonttitle=\sffamily,
    breakable
]}{\end{tcolorbox}}

\definecolor{orange}{rgb}{255, 128, 0}
\newenvironment{lem}[1]{\begin{tcolorbox}[
    arc=0mm,
    colback=white,
    colframe=orange!60!black,
    title={Lemma: #1},
    fonttitle=\sffamily,
    breakable
]}{\end{tcolorbox}}

\definecolor{lightblue}{rgb}{0, 178, 255}
\newenvironment{cor}[1]{\begin{tcolorbox}[
    arc=0mm,
    colback=white,
    colframe=lightblue!60!black,
    title={Corollary: #1},
    fonttitle=\sffamily,
    breakable
]}{\end{tcolorbox}}

\definecolor{peach}{rgb}{255, 0, 172}
\newenvironment{prop}[1]{\begin{tcolorbox}[
    arc=0mm,
    colback=white,
    colframe=peach!60!black,
    title={Proposition: #1},
    fonttitle=\sffamily,
    breakable
]}{\end{tcolorbox}}

\definecolor{lightgrey}{rgb}{64, 64, 64}
\newenvironment{pf}[1]{\begin{tcolorbox}[
    arc=0mm,
    colback=white,
    colframe=lightgrey!60!black,
    title={pf: #1},
    fonttitle=\sffamily,
    breakable
]}{\end{tcolorbox}}

\newenvironment{ex}[1]{\begin{tcolorbox}[
    arc=0mm,
    colback=white,
    colframe=green!60!black,
    title={Ex: #1},
    fonttitle=\sffamily,
    breakable
]}{\end{tcolorbox}}

\newenvironment{remarkbox}{\begin{tcolorbox}[
    arc=0mm,
    colback=white,
    colframe=red!60!black,
    title=Remark,
    fonttitle=\sffamily,
    breakable
]}{\end{tcolorbox}}

\newenvironment{noot}{\begin{tcolorbox}[
    arc=0mm,
    colback=white,
    colframe=white!60!black,
    title=Note,
    fonttitle=\sffamily,
    breakable
]}{\end{tcolorbox}}


% Figure support as explained in my blog post.
\usepackage{import}
\usepackage{xifthen}
\usepackage{pdfpages}
\usepackage{transparent}
\newcommand{\incfig}[1]{%
    \def\svgwidth{\columnwidth}
    \import{./figures/}{#1.pdf_tex}
}

% Fix some stuff
% %http://tex.stackexchange.com/questions/76273/multiple-pdfs-with-page-group-included-in-a-single-page-warning
\pdfsuppresswarningpagegroup=1

\author{Edward Chang}
